\documentclass[12pt]{report}
\usepackage[utf8x]{inputenc}
\usepackage[russian]{babel}
\usepackage{amssymb}
\usepackage{amsmath}
\usepackage[makeroom]{cancel}

\title{Конспекти лекцій з математичного аналізу Анікушина А.В. Модуль 3.}
\author{Автор текста @bezkorstanislav \\ Если есть ошибки, пишите ему в телеграм \\  \small{Aфтар выражает благодарность @vic778 за многочисленные поправки}}

\date{November 2019}

\begin{document}

\maketitle

\begin{center}
\textbf{\LargeДиференціальне числення функції однієї змінної}
\end{center}

\begin{center}
\textbf{\large{Означення похідної. Основні правила диференційонування}}
\end{center}

Нехай $f : \mathbb{R} \to \mathbb{R}$ --- деяка функція однієї змінної, $x_0 \in D_f,\ x_0 \in (D_f)^{'}$ .

\textbf{Означення.} Якщо $\exists \lim\limits_{x \to x_0} \frac{f(x) - f(x_0)}{x - x_0}$, то функція $f$ називається 
\textbf{диференційованою} в точці $x_0$, а сама границя називається \textbf{похідною функції} $f$ в точці $x_0$. І позначається $f'(x_0)$ або $\frac{d f(x_0)}{d x}$.

\vspace{3mm}

\textit{Зауваження 1.} Для функції однієї змінної ми ототожнили диференційованість та існування похідної, що не завжди є правдою для функцій багатьох змінних.

\textit{Зауваження 2.} $$x - x_0 = \Delta x$$
$$f'(x_0) = \lim_{\Delta x \to 0} \frac{f(x_0 + \Delta x) - f(x_0)}{\Delta x}$$
$$f(x_0 + \Delta x) - f(x_0) = \Delta f(x_0)$$
Отже, похідна дорівнює відношенню зміни приросту функції до приросту аргументу, що породжує цей приріст.

\textit{Зауваження 3.} Із теорії границі функції, якщо функція $f$ є диференційованою в точці $x_0$, то:
$$f'(x_0) = \lim_{x \to x_0} \frac{f(x) - f(x_0)}{x - x_0} \Longleftrightarrow \frac{f(x) - f(x_0)}{x - x_0} = f'(x_0) + o(1) \Longleftrightarrow$$
$$\Longleftrightarrow f(x) - f(x_0) = f'(x_0)(x-x_0) + o(x-x_0)$$
Приріст функції є лінійним до приросту аргументу:
$$f(x_0 + \Delta x) - f(x_0) = f'(x_0)\Delta x + o(\Delta x)$$ 
Отже, якщо має місце рівність:
$$f(x) - f(x_0) = A (x - x_0) + o(x - x_0) \Longrightarrow \exists f'(x_0) = A$$

\textbf{Теорема. (Необхідна умова диференційованості).} Функція $f$ є диференційованою в $x_0$ тільки тоді, коли $f$ --- неперервна в точці $x_0$. (Але не завжди неперервна функція є диференційованою)

\textbf{Доведення.} Щоб існувала похідна треба, щоб
$$f(x) - f(x_0) \to 0, x \to x_0$$.
З цього випливає, що $f$ неперервна в точці $x_0$.

\vspace{5mm}

\textbf{Теорема. (Диференційованість композиції функцій).}  Нехай дано функції $f$ і $g$. Точка $x_0 \in D_{f \circ g},\ x_0 \in (D_{f \circ g})'$.

Якщо $g$ диференційована в точці $x_0$, а $f$ диференційована в точці $y_0 = g(x_0)$, то $f \circ g$ диференційована в точці $x_0$ і має місце рівність:
$$(f \circ g)' (x_0) = f'(y_0) g'(x_0)$$

\textbf{Доведення.} 
$$(f \circ g) (x) - (f \circ g) (x_0) = f(g(x)) - f(g(x_0)) = f(y) - f(y_0) =$$ 
$$= f'(y_0)(y - y_0) + o(y - y_0) = f'(y_0)(g(x) - g(x_0)) + o(y - y_0) = $$
$$= f'(y_0)(g'(x_0)(x - x_0) + o(x - x_0)) + o(y - y_0) = $$
$$= f'(y_0)(g'(x_0)(x - x_0) + o(x - x_0)) + o(g(x) - g(x_0)) = $$
$$= f'(y_0)(g'(x_0)(x - x_0) + o(x - x_0)) + o(g'(x_0)(x - x_0) + o(x - x_0))$$

Отже,

$$\lim_{x \to x_0} \frac{(f \circ g)(x) - (f \circ g)(x_0)}{x - x_0} = $$
$$=\lim_{x \to x_0}\frac{f'(y_0)(g'(x_0)(x - x_0)(1 + o(1))) + o(g(x_0)(x - x_0) + o(x - x_0))}{x - x_0} = $$
$$=\lim_{x \to x_0}f'(y_0)(g'(x_0)(1 + o(1))) + o(g(x_0) + o(1)) = $$
$$=\lim_{x \to x_0}f'(y_0)g'(x_0)(1 + o(1)) + o(1) = f'(y_0)g'(x_0)$$

\textbf{Теорема. (Лінійність похідної).} Нехай $f$ і $g$ --- диференційовані в точці $x_0$. Тоді $\forall \alpha,\beta \in \mathbb{R}$:
$$(\alpha f + \beta g)' (x_0) = \alpha f'(x_0) + \beta g'(x_0)$$

\textbf{Доведення.} 

$$\lim_{x \to x_0} \frac{(\alpha f + \beta g)(x) - (\alpha f + \beta g)(x_0)}{x - x_0} = $$
$$\lim_{x \to x_0} \frac{ \alpha f(x) - \alpha f(x_0)}{x - x_0} + \frac{\beta g(x) - \beta g (x_0)}{x - x_0} = $$
$$\lim_{x \to x_0} \alpha \frac{ f(x) - f(x_0)}{x - x_0} + \lim_{x \to x_0}\beta \frac{g(x) - g (x_0)}{x - x_0}  = \alpha f'(x_0) + \beta g'(x_0)$$

\textbf{Теорема. Похідна добутку.} Нехай $f$ і $g$ диференційовані в точці $x_0$. 

Тоді функція $f \cdot g$ теж диференційована в точці $x_0$, причому справедливе наступне співвідношення:
$$(f \cdot g)' (x_0) = f'(x_0)g(x_0) + f(x_0) g'(x_0)$$

\textbf{Доведення.} За означенням:

$$\lim_{x \to x_0} \frac{f(x)g(x) - f(x_0)g(x_0)}{x - x_0} = $$
$$= \lim_{x \to x_0} \frac{f(x)g(x) - f(x)g(x_0) + f(x)g(x_0) - f(x_0)g(x_0)}{x - x_0} = $$
$$= \lim_{x \to x_0} \frac{f(x)(g(x) - g(x_0))}{x - x_0} + \frac{g(x_0)(f(x) - f(x_0))}{x - x_0} = $$

Оскільки $f$ --- диференційована в точці $x_0$, то $\exists \lim\limits_{x \to x_0} \frac{f(x) - f(x_0)}{x - x_0} = f'(x_0)$.

Абсолютно аналогічно $\exists \lim\limits_{x \to x_0} \frac{g(x) - g(x_0)}{x - x_0} = g'(x_0)$.

Окрім того $f$ --- диференційована в точці $x_0$, а значить $f$ --- неперервна в точці $x_0$, а значить:
$$\lim_{x \to x_0} f(x) = f(x_0)$$

Отже:

$$\lim_{x \to x_0} \frac{f(x)(g(x) - g(x_0))}{x - x_0} + \frac{g(x_0)(f(x) - f(x_0))}{x - x_0} = $$
$$= f(x_0) g'(x_0) + g(x_0) f'(x_0)$$

\textbf{Теорема. Похідна частки.} Нехай $f$ і $g$ диференційовані в точці $x_0$ і $g(x_0) \neq 0$.

Тоді, $\frac{f}{g}$ теж диференційована в точці $x_0$ і $\Big(\frac{f}{g} \Big)'(x_0) = \frac{f'(x_0) g(x_0) - f(x_0)g'(x_0)}{g^2(x_0)}$

\textbf{Доведення.} Аналогічно. Це доведення на дз :).

\textit{Приклад.} Обчислити похідну від функції $f$ в точці $x_0$.
\begin{enumerate}

\item $f(x) = x^3$.
$$f'(x_0) = \lim_{x\to x_0}\frac{f(x) - f(x_0)}{x - x_0} = \lim_{x\to x_0}\frac{x^3 - x_0^3}{x - x_0} = \ldots = 3 x_0^2$$

\item $f(x) = \begin{cases} x^3,& x \neq 2 \\ 8, & x = 2\end{cases}$
$$f'(x_0) = \lim_{x \to x_0}\frac{f(x) - f(x_0)}{x - x_0}$$
\begin{enumerate}
\item $x_0 = 2$. $f'(x_0) = \frac{x^3 - 8}{x - 2} = 3 \cdot 4 = 12.$
\item $x_0 \neq 2$. $\lim\limits_{x \to x_0} \frac{x^3 - x_0^3}{x - x_0} = 3 x_0^2$. 
\end{enumerate}

\item $f(x) = \begin{cases} x^3,& x \neq 2 \\ 7, & x = 2\end{cases}$
$$f'(x_0) = \lim_{x \to x_0}\frac{f(x) - f(x_0)}{x - x_0}$$
\begin{enumerate}
\item $x_0 \neq 2$. $\lim\limits_{x \to x_0} \frac{x^3 - x_0^3}{x - x_0} = 3 x_0^2$. 
\item $x_0 = 2$. $f'(x_0) = \lim\limits_{x\to 2}\frac{x^3 - 7}{x - 2} = \lim\limits_{x\to 2}\frac{8 - 7}{2 - 2} = \lim\limits_{x\to 2}\frac{1}{0} = +\infty$.

Похідна не може дорівнювати нескінченності, отже $\nexists f'(2)$. 
\end{enumerate}

\item $f(x) = \begin{cases} x^3,& x \geq 2 \\ 12x - 16, & x < 2\end{cases}$

Як \textbf{не треба} робити:
$$f'(x) = \begin{cases} 3 x^2,& x \geq 2 \\ 12, & x < 2\end{cases}$$
Як \textbf{треба} робити:

\vspace{1mm}

Розглядаємо $3$ випадки:

\begin{enumerate}
\item $x_0 > 2$.

$$\lim_{x \to x_0}\frac{f(x) - f(x_0)}{x - x_0} = \lim_{x \to x_0}\frac{x^3 - x_0^3}{x - x_0} = 3 x_0^2$$

\item $x_0 < 2$.

$$\lim_{x \to x_0}\frac{f(x) - f(x_0)}{x - x_0} = 12$$

\item $x_0 = 2$.

$$\lim_{x \to x_0}\frac{f(x) - f(2)}{x - 2} = \lim_{x \to 2+0}\frac{f(x) - f(2)}{x - 2} = \lim_{x \to 2-0}\frac{f(x) - f(2)}{x - 2} = 12$$
\end{enumerate}

Отже, функція диференційована на всіх дійсній прямій.
 
\end{enumerate}

\textbf{Теорема. (Про диференційованість оберненої функції)}. 
Нехай функція $f: \mathbb{R} \to \mathbb{R}$ --- оборотна, то $x_0 \in D_f$ і $x_0 \in (D_f)',\ y_0 = f(x_0)$. Якщо існує $f'(x_0) \neq 0$ і обернена функція $f^{-1}$ --- неперервна в точці $y_0$, то вона диференційовна в цій точці. Якщо, крім того, $y_0$ --- гранична точка множини $E_f = D_{f^{-1}}$, то 
$$(f^{-1})'(y_0) = \frac{1}{f'(x_0)}$$

\textbf{Доведення.} Лектор не дав:(

\vspace{3mm}

\textit{Зауваженн.} Функція $f$ називається диференційованою на множині $A$ якщо $f$ має 
похідну в кожній точці $x_0 \in A$. Це позначається як $f \in D(A)$, де $D(A)$ --- множина диференційованих на $A$ функцій. 

\begin{center}
\textbf{\largeОдносторонні похідні}
\end{center}

Нехай $x_0 \in (D_f \cap (-\infty, x_0))' \cap (D_f \cap (x_0, +\infty))'\cap D_f$. Це рівносильно тому, що 
$$\exists \{ x_n\}_{n=1}^{\infty} \in D_f : x_n < x_0,\ x_n \to x_0, n \to \infty $$
Границя $\lim\limits_{x\to x_0-}\frac{f(x) - f(x_0)}{x - x_0}$ називається \textbf{лівосторонньою похідною} $f$ в точці $x_0$. Позначаємо її як $f'_{\textnormal{л}}(x_0)$.

Аналогічно, $f'_{\textnormal{п}} (x_0) = \lim\limits_{x \to x_0+} \frac{f(x) - f(x_0)}{x - x_0}$

\vspace{3mm}

\textit{Приклади:}


\begin{enumerate}

\item $f(x) = \begin{cases} 0, & x \leq 0 \\ 1, & x > 1\end{cases}$
$$f'_{\textnormal{л}}(x_0) = \lim_{x \to x_0-} \frac{0 - 0}{x - 0} = 0$$
$$f'_{\textnormal{п}}(x_0) = \lim_{x \to x_0+} \frac{1 - 0}{x - 0} = +\infty \Longrightarrow \nexists f'_{\textnormal{п}}(x_0)$$

\item $f(x) = \begin{cases} 0, & x \leq 0 \\ x, & x > 0\end{cases}$
$$f'_{\textnormal{л}}(x_0) = \lim_{x \to x_0-} \frac{0 - 0}{x - 0} = 0$$
$$f'_{\textnormal{п}}(x_0) = \lim_{x \to x_0+} \frac{x - 0}{x - 0} = 1$$
\end{enumerate}

\textbf{Теорема. (Критерій диференційованості}.) 

Нехай $x_0 \in (D_f \cap (-\infty, x_0))' \cap (D_f \cap (x_0, +\infty))' \cap D_f$. Тоді:
$$\exists f'(x_0) \Longleftrightarrow \exists f'_{\textnormal{л}}(x_0), \ \exists f'_{\textnormal{п}}(x_0),\ f'_{\textnormal{л}}(x_0) = f'_{\textnormal{п}}(x_0)$$

\textbf{Доведення.} 
$$\exists f'(x_0) \Longrightarrow \lim_{x \to x_0}\frac{f(x) - f(x_0)}{x - x_0} \Longleftrightarrow$$
$$\Longleftrightarrow \exists \lim_{x\to x_0+}\frac{f(x) - f(x_0)}{x - x_0},\ \exists \lim_{x\to x_0-}\frac{f(x) - f(x_0)}{x - x_0}, \lim_{x\to x_0+}\frac{f(x) - f(x_0)}{x - x_0} = \lim_{x\to x_0-}\frac{f(x) - f(x_0)}{x - x_0}$$
$$\Longleftrightarrow \exists f'_{\textnormal{п}}(x_0),\ \exists f'_{\textnormal{л}}(x_0),\ f'_{\textnormal{п}}(x_0) = f'_{\textnormal{л}}(x_0)$$


\textit{Зауваження.}
$$f'_{\textnormal{л}}(x_0) \neq \lim_{x\to x_0-}f'(x)$$

\vspace{5mm}

Розглянемо параметрично задану функцію $y(x)$:
$$\begin{cases} y = \varphi(t) \\ x = \Psi(t) \end{cases}$$
$$y(x) = \varphi( \Psi^{-1} (x))$$
Часом виразити $y$ через $x$ може бути просто, як тут:
$$\begin{cases} y = t^3 \\ x = t^5\end{cases} \Longrightarrow y = x^{\frac{3}{5}}$$
Але буває, що не дуже просто:
$$\begin{cases} y = t^3 \\ x = t^5+t\end{cases} \Longleftrightarrow y = ???$$

Але ж \textbf{так} хочеться знайти похідну...

\vspace{3mm}

Припустимо, що $\exists \varphi'$ і $\exists (\Psi^{-1})'$ і $\Psi' \neq 0$.

$$\textrm{Тоді }y_x' = \varphi' ( \Psi^{-1} (x) ) \cdot (\Psi^{-1})' (x) = \varphi'(t) \frac{1}{\Psi'(t)} = \frac{\varphi'(t)}{\Psi'(t)}$$

\textit{Приклад:}

\begin{enumerate}

\item Знайти похідну $y_x$ 
$$\begin{cases} y = t^3 \\ x = t^5\end{cases} \Longleftrightarrow y = x^{\frac{3}{5}}$$

Ми уже знайшли, що $y = x^{\frac{3}{5}}$, отже:
$$y_x' = \frac{3}{5} x^{-\frac{2}{5}} = \frac{3}{5} x^{-\frac{2}{5}} = \frac{3}{5} (t^5)^{-\frac{2}{5}} = \frac{3}{5} t^{-2}$$ 

Але давайте спробуємо \textit{по-нормальному}, застосувавши нашу формулу:
$$y_x' = \frac{(t^3)'}{(t_5)'} = \frac{3}{5} \frac{t^2}{t^4} = \frac{3}{5} t^{-2}$$

\item $$\begin{cases} y = \cos^2 t \\ x = \sin^2 t \end{cases}$$
$$y_x' = \frac{2 \cos t (- \sin t)}{2 \sin t \cos t} = -1$$
Як цікавий факт можна помітити, що $y = 1 - x$, де $x \in [0;1]$


\end{enumerate}


\begin{center}

\textbf{\large{Похідні вищих порядків}}

\end{center}

Кажуть, що порядок похідної є вищим, якщо цей порядок більше за $1$.

Нехай $f \in D((a,b))$. Припустимо, що $f'$ є диференційованою в точці $x_0 \in (a,b)$. Тоді функція $f$ є двічі диференційованою в точці $x_0$, а 
число $(f')'(x_0)$ називається \textbf{другою похідною функції} $f$ в точці $x_0$. Вона позначається як $f''(x_0)$.

\vspace{3mm}

\textit{Наприклад:}
$$f(x) = x^3$$
$$f'(x) = 3 x^2$$
$$f''(x) = (f')'(x) = 6x$$

Аналогічно у функції $f$ існує $n-$та  похідна $f^{(n)}(x)$ на проміжку $(a,b)$ і вона є диференційованою в точці $x_0 \in (a,b)$, того
$f$ має $(n+1)$-шу похідну в точці $x_0$.
$$f^{(n+1)}(x_0) = (f^{(n)})|_{x_0} = \lim_{x\to x_0}\frac{f^{(n)}(x) - f^{(n)}(x_0)}{x - x_0}$$

\textit{Приклад:}
$$f(x) = \ln x$$
$$f'(x) = \frac{1}{x}$$
$$f''(x) = \Big( \frac{1}{x} \Big)' = -\frac{1}{x^2}$$
$$f'''(x) = \Big( -\frac{1}{x^2} \Big)' = 2\frac{1}{x^3}$$
$$f^{(4)}(x) = -3 \cdot 2 \cdot \frac{1}{x^4} = -6 \frac{1}{x^4}$$
$$f^{(n)}(x) = (-1)^{n-1}(n-1)!\frac{1}{x^n}$$

Якщо $f$ має $n$-ту похідну $f^{(n)}(x)$ у кожній точці проміжку $I$, то кажуть, що $f$ є $n$ разів диференційованою і позначається це так:
$$f \in D^{(n)}(I)$$

Якщо при цьому $f^{(n)} \in C(I)$, то кажуть, що $f \in C^{(n)}(I)$.

$$C(I) \supset D(I) \supset C^{(1)}(I) \supset D^{(1)}(I) \supset C^{(2)}(I) \supset \ldots$$

\textit{Домашня робота.} Якщо у функції є похідна, чи обов'язково похідна неперервна?

\vspace{3mm}

Якщо $\forall n \in \mathbb{N}\ \exists f^{(n)}$ на $I$, то кажуть, що $f$ --- нескінченно диференційована на $I$. Позначається так:
$$f \in D^{\infty}(I)$$

\begin{center}

\textbf{\large{Дві властивості $n$-тої похідної}}

\end{center}

\textbf{Теорема. (Лінійність $n$-тої похідної).} Нехай $f,g \in D^{(n)}(I)$, тоді 
$$\forall \alpha, \beta \in \mathbb{R}\ \alpha f  +\beta g \in D^{(n)}(I)$$
$$(\alpha f + \beta g)^{(n)} = \alpha f^{(n)} + \beta g^{(n)}$$

\textbf{Теорема. Формула Лейбніца.} Нехай $f,g \in D^{(n)}(I)$, тоді: 
$$f \cdot g \in D^{(n)}(I)$$
$$(f \cdot g)^{(n)} = \sum_{k = 0}^{n} C_n^k \cdot f^{(n-k)} \cdot g^{(k)}$$

\textbf{Доведення.} Це вам нам Д/З. (Підзказка: воно доводиться так само, як формула бінома Ньютона).

\vspace{3mm}

\textit{Приклад:}
\begin{enumerate}
\item
Обчислити $n$-ту похідну від $f(x) = x^2 \ln x$.

$$(x^2 \ln x)^{(n)} = C_n^0 (x^2)^{(n)} \ln x + C_n^1 (x^2)^{(n-1)} (\ln x)^{(1)} + C_n^2 (x^2)^{(n-2)} (\ln x)^{(2)} + \ldots + $$
$$+ \ldots + C_n^{n-3} (x^2)^{(3)} (\ln x)^{(n - 3)} + C_n^{n-2} (x^2)^{(2)} (\ln x)^{(n - 2)} + C_n^{n-1} (x^2)^{(1)} (\ln x)^{(n - 1)} + $$
$$+ C_n^{n} (x^2)^{(0)} (\ln x)^{(n)}$$

Тепер варто помітити, що:

$$(x^2)^{(1)} = 2x$$
$$(x^2)^{(2)} = 2$$
$$(x^2)^{(3)} = 0$$
$$(x^2)^{(n)} = 0,\ n > 3$$

Отже виходить, що наша сума дорівнює наступному:

$$C_n^{n-2}  \frac{2 (-1)^{n-3}(n-3)!}{x^{n-2}} + C_n^{n-1}  \frac{2x (-1)^{n-2}(n-2)!}{x^{n-1}}+ C_n^{n} \frac{ x^2 (-1)^{n-1}(n-1)!}{x^{n}}$$

\item Обсчислити $(\frac{1}{a-x})^{(n)}$

$$(\frac{1}{a-x})^{(0)} = \frac{1}{a-x}$$
$$(\frac{1}{a-x})^{(1)} = \frac{1}{(a-x)^2}$$
$$(\frac{1}{a-x})^{(2)} = \frac{2}{(a-x)^3}$$
$$(\frac{1}{a-x})^{(n)} = \frac{n!}{(a-x)^{n+1}}$$

\item Обчислити $n$-ту похідну від $f(x) = \frac{1}{x^2 - 3x + 2}$.

$$f(x) = \frac{1}{x^2 - 3x + 2} = \frac{1}{(x-1)(x-2)} = \frac{1}{x-1} \frac{1}{x-2} = \frac{1}{1-x} \frac{1}{2-x} = $$

З формули Лейбніца:

$$f^{(n)} = \sum_{k = 0}^{n} C_n^k \Big( \frac{1}{1-x}\Big)^{(n-k)} \Big( \frac{1}{2-x}\Big)^{(k)}$$

А тепер використовуємо формулу з прикладу $2$.

$$f^{(n)} = \sum_{k = 0}^{n} C_n^k \Big( \frac{(n-k)!}{(1-x)^{n-k+1}}\Big) \Big( \frac{(k!)}{(2-x)^{k+1}}\Big)$$

Цікавий факт: конкретно у цьому завданні можна було зробити набагато простіше:

$$f(x) = \frac{1}{1-x} \cdot \frac{1}{2 - x} = \frac{1}{1-x} - \frac{1}{2-x}$$
$$f^{(n)}(x) = \frac{n!}{(1-x)^{n+1}} - \frac{n!}{(2-x)^{n+1}}$$

\end{enumerate}

\vspace{2mm}

Як обчислити похідну вищого порядку від параметрично заданої функції?

\vspace{3mm}

Ось \textit{приклад}:

$$\begin{cases} y = t^2 \\ x = t^3 + t\end{cases}$$

$$y'_x = \frac{y'_t (t)}{x'_t(t)} = \frac{(t^2)'}{(t^3 + t)'} = \frac{2t}{3t^2 + 1}$$

Початкова система задавала залежність $y$ від $x$. Тепер ми можемо побудувати систему залежності $y'$ від $x$:

$$\begin{cases} y_x'(t) = \frac{2t}{3t^2 + 1} \\ x = t^3 + t\end{cases}$$

$$y_x'' = (y_x')' = \frac{(y_x')_t'}{x_t'} = \frac{\Big( \frac{2t}{3t^2 + 1} \Big)'}{(t^3 + t)'} = \frac{-6t^2 + 2}{(3t^2 + 1)^3}$$

Тепер ми можемо побудувати залежність $y_x''$ від $x$:

$$\begin{cases} y_x''(t) =  \frac{-6t^2 + 2}{(3t^2 + 1)^3} \\ x = t^3 + t\end{cases}$$

І так далі...

\begin{center}
\textbf{\large Диференціал функції}
\end{center}

Функція $f$ називається диференційованою в точці $x_0$, якщо:

$$\exists \lim_{x \to x_0}\frac{f(x) - f(x_0)}{x - x_0}$$

Нехай $x - x_0 = \Delta x$. Тоді:

$$\exists \lim_{x \to x_0}\frac{f(x) - f(x_0)}{x - x_0} = \lim_{\Delta x \to 0} \frac{f(x_0 + \Delta x) - f(x_0)}{\Delta x} = f'(x)$$

$$\frac{f(x_0 + \Delta x) - f(x_0)}{\Delta x} = f'(x_0) + o(1)$$

Домножимо все на $\Delta x$:

$$f(x_0 + \Delta x) - f(x_0) = \Delta x \cdot f'(x_0) + o(\Delta x)$$

Отже, $\Delta x f'(x_0)$ --- головна частина приросту функції.

\textbf{Означення.} Лінійне відображення 
$$L(h) = f'(x_0) \cdot h$$
Називається \textbf{диференціалом функції} $f$ в точці $x_0$. Це позначається так:
$$d_{x_0}f(h) = f'(x_0) \cdot h$$

\textit{Приклад:}

$$f(x) = x^3,\ x_0 = 1$$
$$f'(x_0) = 3$$
$$d_{x_0}f(h) = 3h$$

\textit{Геометричний зміст диференціалу.} Диференціал задає рівняння прямої, яка паральна дотичній до графіку функції у точці $x_0$ і проходить через початок координат.

\vspace{5mm}

Розглянемо $f(x) = x$. Тоді:
$$f'(x_0) = 1,\ \textrm{для } \forall x_0 \in \mathbb{R}$$
Тоді:
$$d_{x_0}x(h) = 1 \cdot h = h$$

А тепер нехай $g$ --- довільна диференційована в $x_0$ функція:

$$(d_{x_0}g)(h) = g'(x_0) \cdot h = g'(x_0) \cdot d_{x_0} x(h)$$

Якщо ми приберемо $h$, отримуємо:

$$d_{x_0} g = g'(x_0) \cdot d_{x_0} x$$

Запис, який ще часто можна зустріти:

$$dg = g'(x_0)dx$$ 

\textit{Зауваження.} Диференціал функції $f$ у точці $x_0$ $d_{x_0} f$ ще позначають $df|_{x_0},\ df(x_0)$.

Інколи точку $x_0$ взагалі не пишуть і пишуть просто $df$.

\vspace{3mm}

\textit{Зауваження.} З теореми про лінійність похідної, похідної частки, добутку та суперпозиції випливають наступні правила дій з диференціалом:
$$d(f \pm g) = df \pm dg$$
$$d(f \cdot g) = df \cdot g + f \cdot dg$$
$$d\Big(\frac{f}{g}\Big) = \frac{df \cdot g - f \cdot dg}{g^2}$$
$$d(f(g(x))) = f'(g(x)) \cdot dg(x)$$
Ці властивості справедливі при виконанні умов відповідних теорем. 

\textit{Приклад:}
$$f = \ln (e^{u^2} + v)$$
$$df = d\ln (e^{u^2} + v) = \frac{1}{e^{u^2} + v} \cdot d(e^{u^2} + v) = \frac{d e^{u^2} + dv}{e^{u^2} + v} = $$
$$= \frac{e^{u^2} \cdot d u^2 + dv}{e^{u^2} + v} = \frac{e^{u^2} \cdot 2 \cdot u \cdot du + dv}{e^{u^2} + v}$$

\textit{Домашнє завдання.} Познайомитися з поняттям подвійної похідної та властивостями другого диференціалу. 

(Лектор сказав, що такого на іспиті не буде і взагалі ця тема буде використана у другому семетрі, тож не зробиши цього багато не втратите.)

\begin{center}
\textbf{\large Теорема про середні}
\end{center}

\textit{Означення.} Нехай $x_0 \in D'_f \cap D_f$. 

Функція $f$ має \textbf{локальний максимум} в точці $x_0$, якщо $\exists$ деяких $\varepsilon$-окіл точки $x_0$ $U_{\varepsilon} (x_0 - \varepsilon, x_0 + \varepsilon)$:
$$\forall x \in (U_{\varepsilon} \cap D_f) \ f(x) \leq f(x_0)$$
Аналогічно означається локальний мінімум.

\textit{Означення.} Якщо в точці $x_0$ $f$ має локальний максимум і при цьому
$$\forall x \in (U_{\varepsilon} \cap D_f) \setminus \{ x_0\} \ f(x) < f(x_0)$$
Тоді кажуть, що в точці $x_0$ $f$ має \textbf{строгий локальний максимум}.

\textbf{Теорема Ферма.} Нехай $x_0 \in (D_f \cap (-\infty, x_0))' \cap (D_f \cap (x_0, +\infty))' \cap D_f$.

Якщо $f$ має локальний екстремум (тобто мінімум або максимум) в точці $x_0$ і є диференційованою в точці $x_0$, то $f'(x_0) = 0$.

\textbf{Доведення.} Згідно критерію диференційованості:
$$\exists f'(x_0) \Longleftrightarrow \exists f'_{\textnormal{л}}(x_0), f'_{\textnormal{п}}(x_0), f'_{\textnormal{л}}(x_0) = f'_{\textnormal{п}}(x_0)$$
$$\lim_{x \to x_0-}\frac{f(x) - f(x_0)}{x - x_0} = \lim_{x \to x_0+}\frac{f(x) - f(x_0)}{x - x_0}$$
Розглянемо випадок, коли $f$ має локальний мінімум. Випадок з максимумом доводиться аналогічно.

Якщо $f$ має локальний мінімум в точці $x_0$, то:
$$\forall x \in U_{\varepsilon} \cap D_f$$
$$f(x_0) \leq f(x) \Longrightarrow f(x) - f(x_0) \geq 0$$
При $x \to x_0-$:
$$x - x_0 < 0$$
$$f(x) - f(x_0) \geq 0,\ \textrm{отже:}$$
$$\frac{f(x) - f(x_0)}{x - x_0} \leq 0, \textrm{для } \forall x \in (U_{\varepsilon} \cap D_f),\ \textrm{отже:}$$
$$\lim_{x\to x_0-} \frac{f(x) - f(x_0)}{x - x_0} \leq 0$$
Аналогічно 
$$\lim_{x\to x_0+} \frac{f(x) - f(x_0)}{x - x_0} \geq 0$$
$$0 \geq \lim_{x\to x_0-} \frac{f(x) - f(x_0)}{x - x_0} = \lim_{x\to x_0+} \frac{f(x) - f(x_0)}{x - x_0} \geq 0$$
Отже:
$$\lim_{x\to x_0-} \frac{f(x) - f(x_0)}{x - x_0} = \lim_{x\to x_0+} \frac{f(x) - f(x_0)}{x - x_0} = 0 \Longrightarrow$$
$$f'(x_0) = 0$$

\textit{Зауваження.} Зворотнє твердження не є правдивим. Наприклад функція $f = x^3$ не має екстремуму в $x_0 = 0$, але $f'(0) = 0$.

\textbf{Теорема Ролля.} Нехай $f \in C([a,b])\cap D((a,b))$.
$$\textrm{Якщо } f(a) = f(b), \textrm{ то } \exists \xi \in (a,b) : f'(\xi) = 0$$
\textbf{Доведення.} Використаємо теорему Вейершрасса. Тоді $f$ досягає на $[a,b]$ свого максимального і мінімального значення. Нехай
$$f(x^*) = \max_{x \in [a,b]}f(x)$$
$$f(x_*) = \min_{x \in [a,b]}f(x)$$
Якщо хоча б одне з цих чисел $x^*, x_*$ потрапляє на інтервал $(a,b)$, то виконано всі умови теореми Ферма, а отже $f'$ в цій точці дорівнює нулю за теоремою Ферма.

Якщо обидві точки не належать інтервалу $(a,b)$, то $\{ x^*, x_*\} \subset \{ a,b\}$.

Зважаючи, що $f(a) = f(b)$, тоді 
$$f(x^*) = f(x_*) \Longrightarrow f \textrm{ є сталою на $[a,b]$}$$
Тоді $\forall \xi \in (a,b)\ f'(\xi) = 0$

\vspace{3mm}

\textit{Приклад:}
Нехай $f \in D([0,1])$ і $f(0) = f(1) = 0$. Тоді $\forall \alpha \in \mathbb{R}$ рівняння 
$$f'(x) + \alpha f(x) = 0$$
Має розв'язок на $[0,1]$.
\textit{Доведення.} Розглянемо функцію
$$F(x) = e^{\alpha x}f(x)$$
$$F(0) = 0,\ F(1) = 0$$
Тоді за теоремою Ролля $\exists \xi \in [a,b]$:
$$F'(\xi) = 0$$
$$F'(x) = e^{\alpha x}\cdot f'(x) + \alpha e^{\alpha x} \cdot f(x)$$
$$e^{\alpha \xi}\cdot f'(\xi) + \alpha e^{\alpha \xi} \cdot f(\xi) = 0$$
Поділимо вираз на $e^{\alpha \xi}$.
$$f'(\xi) + \alpha f(\xi) = 0$$

\vspace{3mm}

\textbf{Теорема Дарбу}. Нехай $f \in C([a,b])\cap D((a,b))$ і $f'_{\textnormal{п}}(a) \cdot f'_{\textnormal{л}}(b) < 0$. Тоді $\exists \xi : f'(\xi) = 0$.

\textbf{Доведення.} Використаємо теорему Вейершрасса. Тоді $f$ досягає на $[a,b]$ свого максимального і мінімального значення. Нехай
$$f(x^*) = \max_{x \in [a,b]}f(x)$$
$$f(x_*) = \min_{x \in [a,b]}f(x)$$
Якщо хоча б одне з цих чисел $x^*, x_*$ потрапляє на інтервал $(a,b)$, то виконано всі умови теореми Ферма, а отже $f'$ в цій точці дорівнює нулю за теоремою Ферма.

Якщо обидві точки не належать інтервалу $(a,b)$, то $\{ x^*, x_*\} \subset \{ a,b\}$. 

Тепер у нас можливі два випадки:
\begin{itemize}

\item Якщо $x^* = x_*$, то за теоремою Ролля все працює.
\item Якщо $x^* \neq x_*$.

Нехай $x^* = a,\ x_* = b$, тоді:

$$f'_{\textnormal{п}}(a) = \lim_{x \to a+}\frac{f(x) - f(a)}{x - a}$$
$$x - a > 0,\ f(x) - f(a) \leq 0 \Longrightarrow $$
$$f'_{\textnormal{п}}(a) = \lim_{x \to a+}\frac{f(x) - f(a)}{x - a} \leq 0$$
Аналогічно
$$f'_{\textnormal{л}}(b) = \lim_{x \to b-}\frac{f(x) - f(b)}{x - b} \leq 0$$
Виходить, що:
$$f'_{\textnormal{п}}(a) \cdot f'_{\textnormal{л}}(b) \geq 0$$
Що суперечить умові.

\end{itemize}

\vspace{5mm}

\textbf{Теорема про проміжні значення похідної.} Нехай $f \in D([a,b])$. Тоді $f'$ приймає всі значення між $f'_{\textnormal{п}}(a)$ та $f'_{\textnormal{л}}(b)$.

\textbf{Доведення.} Нехай $\alpha$ --- дійсне число між $f'_{\textnormal{п}}(a)$ та $f'_{\textnormal{л}}(b)$. Розглянемо
$$F(x) = f(x) - \alpha \cdot x$$
$$F \in D([a,b])$$
$$F'_{\textnormal{п}}(a) = f'_{\textnormal{п}}(a) - \alpha < 0$$ 
$$F'_{\textnormal{л}}(b) = f'_{\textnormal{л}}(b) - \alpha > 0$$
Отже, ми точно знаємо, що
$$F'_{\textnormal{п}}(a) \cdot F'_{\textnormal{л}}(b) < 0$$
Отже, всі умови теореми Дарбу виконано, отже:
$$\exists \xi : F'(\xi) = 0 \Longrightarrow f'(\xi) - \alpha = 0 \Longrightarrow f'(\xi) = \alpha$$


\textbf{Теорема Лагранжа.} Нехай $f \in C([a,b]) \cap D((a,b))$. 
$$\textrm{Тоді } \exists \xi \in (a,b):$$
$$f'(\xi) = \frac{f(b) - f(a)}{b - a}$$
\textbf{Доведення.} Розглянемо функцію 
$$F(x) = f(x) - \frac{f(b) - f(a)}{b - a}x$$
$$F \in C([a,b])\cap D((a,b))$$
$$F(a) = f(a) - \frac{f(b) - f(a)}{b - a} \cdot a = $$
$$\frac{bf(a) - a f(a) - a f(b) + f(a) a}{b - a} = $$
$$\frac{b f(a) - a f(b)}{b - a}$$
Аналогічними перетвореннями отримуємо
$$F(b) = \frac{-f(b) a + f(a) b}{b - a}$$
$$\textrm{Виходить, що }F(a) = F(b)$$
Отже, за теоремою Ролля $\exists \xi \in (a,b) : F'(\xi) = 0$
$$F'(\xi) = f'(\xi) - \frac{f(b) - f(a)}{b - a} = 0 \Longrightarrow$$
$$f'(\xi) = \frac{f(b) - f(a)}{b - a}$$

\textit{Наслідок.} Нехай $f \in D([a,b])$ і $\forall x \in (a,b) \ f'(x) = 0$, тоді 
$$f = const \textrm{ на $(a,b)$}$$

\textit{Доведення.} $\forall x_1, x_2 \in (a,b)$.

Використаємо теорему Лагранжа для $[x_1, x_2]$
$$\textrm{Тоді } \exists \xi \in (x_1, x_2):$$
$$\frac{f(x_1) - f(x_2)}{x_1 - x_2} = f'(\xi)$$
$$f'(\xi) = 0 \Longrightarrow f(x_1) - f(x_2) = 0 \Longrightarrow f(x_1) = f(x_2)$$

\textit{Приклад.} Довести, що для $\forall x_1, x_2 \in \mathbb{R}$ 
$$|\arctg x - \arctg y| \leq |x-y|$$

$f(x) = \arctg x$ задовольняє усі умови теореми Лагранжа на $[x,y]$. Тоді:
$$\exists \xi \in (x,y):$$
$$\frac{\arctg x - \arctg y}{x - y} = (\arctg x)'|_{x = \xi} = \frac{1}{1 + \xi^2}$$
$$\Big| \frac{\arctg x - \arctg y}{x - y}\Big| = \frac{1}{1 + \xi^2} \leq 1$$
Домноживши обидві частини на $|x - y|$ отримаємо:
$$|\arctg x -\arctg y| \leq |x-y|$$

\textit{Геометрична інтерпретація теореми Лагранжа.} Якщо $f \in C([a,b]) \cap D((a,b))$, то завжди можна провести дотичну до функції $f$, паралельну до відрізка, що сполучає точки $(a, f(a))$ та $(b, f(b))$.

\vspace{3mm}

\textbf{Теорема Коші.}  
Нехай $f, g \in C([a,b]) \cap D((a,b))$. Тоді $\exists \xi \in (a,b):$
$$(f(b) - f(a))g'(\xi) = (g(a) - g(b))f'(\xi)$$
\textbf{Доведення.} Розглянемо допоміжну функцію 
$$F(x) = (f(b) - f(a))g(x) - (g(b) - g(a))f(x)$$
\begin{enumerate}
\item $F \in C([a,b])\cap D((a,b))$
\item $F(a) = (f(b) - f(a))g(a) - (g(b) - g(a))f(a) = f(b)g(a) - f(a)g(a) + f(a)g(a) - g(b)f(a) = f(b)g(a) - g(b)f(a)$
	
\vspace{1mm}

$F(b) = (f(b) - f(a))g(b) - (g(b) - g(a))f(b) = f(b)g(b) - f(a)g(b) + f(b)g(a) - g(b)f(b) = f(b)g(a) - g(b)f(a)$
Отже, $F(a) = F(b)$.
\end{enumerate}

Тоді за теоремою Ролля 
$$\exists \xi : F'(\xi) = 0$$
$$(f(b) - f(a))g'(\xi) - (g(b) - g(a))f'(\xi) = 0 \Longleftrightarrow$$
$$(f(b) - f(a))g'(\xi) = (g(b) - g(a))f'(\xi) $$

\textit{Наслідок.} Якщо в умовах теореми Коші $\forall x \in (a,b) \ g'(x) \neq 0$, то
$$\frac{f(b) - f(a)}{g(b) - g(a)} = \frac{f'(\xi)}{g'(\xi)}$$
Чому у нас не буде раптом ділення на нуль?

Очевидно, що з $\forall x \in (a,b) \ g'(x) \neq 0$ слідує те, що $g'(\xi) \neq 0$.

Подивимось на $g(b) - g(a)$. Якби $g(b) - g(a) = 0$, то за теоремою Ролля $\exists c \in (a,b) : g'(c) = 0$, що суперечить умові.

\textbf{Теорема про неперервність похідної.} Нехай $f \in (D((a,b))\setminus \{x_0\})$, $f \in C((a,b))$, $x_0 \in (a,b)$.

$$\textrm{Якщо } \exists \lim_{x \to x_0}f'(x) = \lambda \Longrightarrow \exists f'(x_0) = \lambda$$

\textbf{Доведення.} Розглянемо проміжки $[x_0, x]$, $[x, x_0]$.

\begin{itemize}
\item Розглянемо $[x_0, x]$. За теоремою Лагранжа:
$$\exists \xi \in [x_0, x]: f'(\xi) = \frac{f(x) - f(x_0)}{x - x_0}$$
$$x_0 < \xi < x$$
Тож при $x \to x_0$, $\xi \to x_0$. Отже:
$$f'(\xi) \to \lambda \Longrightarrow$$
$$\Longrightarrow \exists \lim_{x \to x_0+}\frac{f(x) - f(x_0)}{x - x_0} = \lambda$$
\item Аналогічно, розглядаючи проміжок $[x, x_0]$ доводимо, що 
$$\exists \lim_{x\to x_0 - }\frac{f(x) - f(x_0)}{x - x_0} = \lambda$$

\end{itemize}
\textbf{Теорема. Типи розривів похідної.}

Нехай $f \in D((a,b))$ тоді $\forall x \in (a,b)$.

$f'$ або є неперервною в точці $x$ або має розрив другого роду.

\textbf{Доведення.} Розглянемо
$$f'(x_0+0),\ f'(x_0-0)$$
\begin{enumerate}
\item Якщо $f'(x - 0)$ або $f'(x + 0)$ не існує або дорівнює нескінченності, то $f'$ має розрив другого роду.

\item Якщо $\exists f'(x - 0)$ і $\exists f'(x + 0)$.

Тоді за попередньою теоремою
$$\exists f_{\textrm{л}}'(x) = \lim_{x\to x_0-}f'(x),\ \exists f_{\textrm{п}}'(x) = \lim_{x\to x_0+}f'(x)$$

\begin{itemize}
\item Якщо $f_{\textrm{л}}'(x) \neq f_{\textrm{п}}'(x)$. Тоді $\nexists f(x_0)$. Протиріччя.
\item Якщо $f_{\textrm{л}}'(x) = f_{\textrm{п}}'(x) \Longrightarrow \exists \lim\limits_{x\to x_0} f'(x)$. Тоді за попередньою теоремою 
$\exists f'(x) = \lim\limits_{x \to x_0}f'(x)$. Отже, $f'$ --- неперервна в точці $x_0$. 
\end{itemize}
\end{enumerate}

\textit{Приклад.} 
$$f(x) = \begin{cases} x^2 \sin \frac{1}{x}, & x \neq 0 \\
				0, x = 0\end{cases}$$
При $x \neq 0$:
$$f'(x) = 2x \sin \frac{1}{x} - \cos \frac{1}{x},\ x\neq 0$$
Інакше:
$$f'(0) = \lim_{x \to 0}\frac{f(x) - f(0)}{x - 0} = \lim_{x\to 0}\frac{x^2 \sin \frac{1}{x}}{x} = $$
$$= \lim_{x\to 0}x \sin \frac{1}{x} = 0$$
Чи буде $f'$ неперервна в точці $x_0$?
$$\lim_{x \to 0}f'(x) = \lim_{x\to 0 }(2x \sin \frac{1}{x} - \cos \frac{1}{x}) = -\lim_{x\to 0}\cos \frac{1}{x} = $$
$$-\lim_{y \to \infty} \cos y \textrm{ --- не існує}$$
Отже $f'(x)$ має розрив другого роду у точці $0$.

\textbf{Правило Лопіталя.} Нехай $f,g : (a,b) \to \mathbb{R}$.

\textbf{Теорема.} Нехай $f,g \in D((a,b))$ і прицьому:
\begin{enumerate}
\item $$\lim_{x\to a+}f(x) = \lim_{x \to a+}g(x) = 0$$ 
\item
$$\exists \lim_{x\to a+}\frac{f'(x)}{g'(x)} = \lambda \in \mathbb{R}$$
\item 
$$g'(x) \neq 0 \ \forall x \in (a,b)$$
\end{enumerate}
$$\textrm{Тоді } \exists \lim_{x\to a+}\frac{f(x)}{g(x)} = \lambda$$

\textbf{Доведення.} Доозначимо функції $f$ і  $g$ у точці $a$. Розглянемо 
$$F(x) = \begin{cases} 0,& x = a \\
						f(x),& x \in (a,b)\end{cases}$$
Аналогічно
$$G(x) = \begin{cases} 0,& x = a \\
						g(x),& x \in (a,b)\end{cases}$$
$F$ та $G$ задовольняють умови наслідку з теореми Коші на проміжку $[a,x]$, $x \in (a,b)$. Тоді $\exists \xi:$
$$\frac{F(x) - F(a)}{G(x) - G(a)} = \frac{F'(\xi)}{G'(\xi)} \Longrightarrow \frac{f(x)}{g(x)} = \frac{f'(\xi)}{g'(\xi)}$$
$$x \to a+,\ a < \xi < x, \Longrightarrow \xi \to a+ \Longrightarrow $$
$$\Longrightarrow \exists \lim_{\xi \to a+}\frac{f'(\xi)}{g'(\xi)} \Longrightarrow \exists \lim_{x\to a+}\frac{f(x)}{g(x)} = \lambda$$
\textit{Зауважте,} що слідування йде тільки в одну сторону.

\vspace{3mm}

Аналогічна теорема при $x \to +\infty$:
\textbf{Теорема.} $f,g : [a, +\infty] \to \mathbb{R},\ f,g \in D((a, +\infty))$.
\begin{enumerate}
\item $$\exists \lim_{x \to +\infty}f(x) = \lim_{x\to +\infty}g(x) = 0$$
\item $$\exists \lim_{x\to +\infty}\frac{f'(x)}{g'(x)} = \lambda$$
\item $$\forall x \in (a, +\infty)\ g'(x) \neq 0$$
\end{enumerate}  
$$\textrm{Тоді}\ \exists \lim_{x\to +\infty}\frac{f(x)}{g(x)} = \lambda$$

\textit{Зауваження.} Умову $\lim f = \lim g = 0$ можна замінити на $\lim f = \lim g = \infty$ і правило Лопіталя теж буде виконуватись. Головне, щоб функції прямували до $0$ або $\infty$ одночасно.

\textit{Приклад:}
\begin{itemize}
\item $$\lim_{x\to 0}\frac{\sin x}{e^x - 1} = \lim_{x\to 0}\frac{\cos x}{e^x} = 1$$
\item $$\lim_{x\to +\infty}\frac{e^x}{x} = \lim_{x \to +\infty}\frac{e^x}{1} = +\infty$$
\item $$\lim_{x\to 0}x \ln x = \lim_{x \to 0}\frac{\ln x}{\frac{1}{x}} = \lim_{x\to 0}\frac{\frac{1}{x}}{\frac{-1}{x^2}} = \lim_{x\to 0}(-x) = 0$$
\end{itemize}

\begin{center}
\textbf{Формула Тейлора}
\end{center}

Нехай $f : (a,b) \to \mathbb{R},\ D_f = (a,b)$.

Многочлен

$$P_{n,x}(x_0) = \sum_{k = 0}^{n}\frac{f^{(k)}(x_0)}{k!} \cdot (x-x_0)^k = $$
$$= f(x_0) + \frac{f'(x_0)}{1!} + \frac{f''(x_0)}{2!}(x-x_0)^2 + \ldots + \frac{f^{(n)}(x_0)}{n!} (x-x_0)^n$$

називається \textbf{многочленом Тейлора} $n$-того порядку для функції $f$ в точці $x_0$.

$$r(x) = f(x) - P(x)$$
Називається \textbf{залишковим членом}.

$$f(x) = P(x) + r(x)$$

\textbf{Теорема. Формула Тейлора із залишкоим членом у формі Пеано}.

Нехай $f\in D^{(n-1)}((a,b)) \cap D^{(n)}(\{ x_0\}),\ x_0, x \in (a,b)$. Тоді при $x \to x_0$:

$$r(x) = o((x - x_0)^n)$$

Тобто:

$$f(x) = \sum_{k = 0}^{n}\frac{f^{(k)}(x_0)}{k!} \cdot (x-x_0)^k + o((x - x_0)^n)$$

\textbf{Доведення.}

$$\lim_{x\to x_0}\frac{f(x) - \sum_{k = 0}^{n}\frac{f^{(k)}(x_0)}{k!} \cdot (x-x_0)^k}{(x - x_0)^n} = $$
$$ = \lim_{x\to x_0}\frac{(f(x) - \sum_{k = 0}^{n}\frac{f^{(k)}(x_0)}{k!} \cdot (x-x_0)^k)'}{((x - x_0)^n)'} $$

$$\Big(\frac{f^{(k)}(x_0)}{k!} (x - x_0)^k\Big)' = \frac{f^{(k)}(x_0)}{(k-1)!} (x - x_0)^{k-1}\textrm{, отже:}$$
$$\lim_{x\to x_0}\frac{(f(x) - \sum_{k = 0}^{n}\frac{f^{(k)}(x_0)}{k!} \cdot (x-x_0)^k)'}{((x - x_0)^n)'} = $$
$$= \lim_{x \to x_0}\frac{f'(x) - f'(x_0) - \frac{f''(x_0)}{1!}(x - x_0) - \ldots - \frac{f^{(n)}(x_0)}{(n-1)!} (x - x_0)^{n-1}}{n(x - x_0)^{n-1}} = $$
$$= \lim_{x \to x_0}\frac{(f'(x) - f'(x_0) - \frac{f''(x_0)}{1!}(x - x_0) - \ldots - \frac{f^{(n)}(x_0)}{(n-1)!} (x - x_0)^{n-1})'}{(n(x - x_0)^{n-1})'} = $$
$$= \ldots = \lim_{x \to x_0}\frac{f^{n-1}(x) - f^{n-1}(x_0) - \frac{f^{n}(x_0)}{1} (x - x_0)}{n! (x - x_0)} = $$
$$= \frac{1}{n!} \lim_{x\to x_0}\Big( \frac{f^{(n-1)}(x) - f^{(n-1)}(x_0)}{x - x_0} - f^{(n)}(x_0)\Big)$$
Оскільки $f \in D^{(n+1)}(\{ x_0\})$, то:
$$\exists f^{(n)}(x_0) = \lim_{x\to x_0} \frac{f^{(n-1)}(x) - f^{(n-1)}(x_0)}{x - x_0}$$
Отже: 
$$\frac{1}{n!} \lim_{x\to x_0}\Big( \frac{f^{(n-1)}(x) - f^{(n-1)}(x_0)}{x - x_0} - f^{(n)}(x_0)\Big) = $$
$$\frac{1}{n!} \lim_{x \to x_0}(f^{(n)}(x_0) - f^{(n)}(x_0)) = 0$$

Доведено.

\vspace{3mm}

\textbf{Теорема. Формула Тейлора із залишковим членом у формі Лагранжа.}
Нехай $f \in D^{(n+1)}((a,b)),\ x,x_0 \in (a,b)$, тоді:
$$\exists \xi \in (x_0, x):$$
$$r(x) = \frac{f^{(n+1)}(\xi)}{(n+1)!}\cdot (x - x_0)^{n+1}$$
Тобто
$$f(x) = \sum_{k = 0}^{n}\frac{f^{(k)}(x)}{k!} \cdot (x-x_0)^k + \frac{f^{(n+1)}(\xi)}{(n+1)!}\cdot (x - x_0)^{n+1}$$

\textbf{Теорема. Формула Тейлора із залишковим членом у формі Коші.}
$$\exists \xi \in (x, x_0):$$
$$r(x) = \frac{\theta^n}{n!}(x - x_0)^n f^{(n+1)}(\xi)$$
Що таке $\theta$? 
Це деяке число, яке належить проміжку $[0,1]$. Завдяки цьому виходить, що:
$$\theta x_0 + (1-\theta)x \in [x, x_0]$$
Оскільки $\xi \in (x, x_0)$, то:
$$\xi = \theta x_0 + (1 - \theta x),\ \theta \in (0,1)$$

\textit{Домашнє завдання:}
\begin{enumerate}
\item Виразити $\theta$ через $\xi$ і позбавити форму Коші від неї.
\item Прочитати доведення цих двух теорем.
\item Послабити умову теореми.
\end{enumerate}

\vspace{3mm}

\textit{Приклад:}
При $x_0 = 0$:
$$\sin x = x - \frac{x^3}{3!} + r(x),\ x \in [0,1]$$
$$\sin x = x - \frac{x^3}{3!} + \frac{\sin^{(4)} (\xi)}{4!} x^4 = x - \frac{x^3}{3!} + \frac{\sin \xi}{4!} x^4$$
$$\Big|\sin x - (x - \frac{x^3}{3!})\Big| = \Big|\frac{\sin \xi}{4!} x^4\Big|\leq \frac{1}{24}$$

\begin{center}
\textbf{Дослідження функцій за допомогою похідних}
\end{center}

Нехай $f : (a,b) \to \mathbb{R},\ D_f = (a,b)$.

\textbf{Теорема.} Нехай $f \in D((a,b))$. Тоді:
\begin{enumerate}
\item $f \nearrow$ (нестрого зростає) на $(a,b)$ $\Longleftrightarrow$ $\forall x \in (a,b)\ f'(x) \geq 0$. 
\item $f \searrow$ (нестрого спадає) на $(a,b)$ $\Longleftrightarrow$ $\forall x \in (a,b)\ f'(x) \leq 0$.
\item $f \uparrow$ (строго зростає) на $(a,b)$ $\Longleftarrow$ $\forall x \in (a,b)\ f'(x) > 0$.
\item $f \downarrow$ (строго спадає) на $(a,b)$ $\Longleftarrow$ $\forall x \in (a,b)\ f'(x) < 0$.
\end{enumerate}

\textbf{Доведення.} 
\begin{enumerate}
\item $\Longrightarrow$. $\forall x_0 \in (a,b)$:
$$f'(x) = \lim_{x \to x_0}\frac{f(x) - f(x_0)}{x - x_0}$$
Оскільки функція нестрого зростає, то при $x > x_0$:
$$f(x) - f(x_0) \geq 0,\ x - x_0 > 0 \Longrightarrow \lim_{x \to x_0+}\frac{f(x) - f(x_0)}{x - x_0} \geq 0$$
Аналогічно:
$$\lim_{x \to x_0-}\frac{f(x) - f(x_0)}{x - x_0} \geq 0$$
Отже й $f'(x) \geq 0$.

\vspace{2mm}

$\Longleftarrow$. $\forall x_1, x_2 \in (a,b)$:
Нехай $x_1 < x_2$. За теоремою Лагранжа можна записати наступне:
$$f(x_2) - f(x_1) = f'(\xi)(x_2 - x_1)$$
$$\textrm{Оскільки } f'(\xi) \geq 0,\ x_2 - x_1 > 0 \Longrightarrow f(x_2) - f(x_1) = f'(\xi)(x_2 - x_1) \geq 0$$
Отже, при $x_1 < x_2$ $f(x_2) - f(x_1) \geq 0 \Longleftrightarrow f(x_2) \geq f(x_1)$, що й треба було довести.

\item Доводиться абсолютно аналогічно з попереднім.

\item $\Longleftarrow$. Доводиться аналогічно з $1$. Але, обернена тверждення ($\Longrightarrow$) тепер не є правдою, бо, наприклад:
$$f(x) = x^3\textrm{--- строго зростнає на $\mathbb{R}$, але:}$$
$$f'(x) = 2x^2 \Longrightarrow f'(0) = 0$$
Отже, навіть якщо функція строго зростає, $f'$ все одно може набувати нульове значення.
\item Абсолютно аналогічно з попереднім.
\end{enumerate}

\textbf{Означення.} Функція $f$ називається зростаючою в точці $x_0$, якщо:
$$\exists \varepsilon > 0$$
$$\forall x \in (x_0, x_0 + \varepsilon)\ f(x) > f(x_0)$$
$$\forall x \in (x_0 - \varepsilon, x_0)\ f(x) < f(x_0)$$

Аналогічно означується спадна, неспадна та зростаюча функції у точці

\vspace{3mm}

\textbf{Теорема.} Якщо $f'(x_0) > 0$, то функція зростає в точці $x_0$.

\textbf{Доведення.} 
$$f'(x_0) = \lim_{x \to x_0}\frac{f(x) - f(x_0)}{x - x_0} > 0 \Longrightarrow$$
$$\exists \varepsilon > 0 : \forall x \in (x_0 - \varepsilon, x_0 + \varepsilon)\ \frac{f(x) - f(x_0)}{x - x_0} > 0$$
Отже:
$$x < x_0 \Longrightarrow f(x) - f(x_0) < 0 \Longleftrightarrow f(x) < f(x_0)$$
$$x > x_0 \Longrightarrow f(x) - f(x_0) > 0 \Longleftrightarrow f(x) > f(x_0)$$

\begin{center}

\textbf{\largeДоведення рівностей та нерівностей}

\end{center}

\begin{enumerate}
\item 
Довести, що $\forall a,b \ a^2 + b^2 = (a-b)(a+b)$.

Розглянемо функцію $f(x) = x^2 - b^2 - (x-b)(x+b)$. Тепер нам треба довести, що $\forall x\ f(x) = 0$.
$$f'(x) = 2x - (x+b + x - b) = 2x - 2x = 0 \Longrightarrow f(x) = const \in \mathbb{R}$$
Тепер нам достатньо перевірити значення функції в будь-якій точці. Нехай це буде точка $b$:
$$f(b) = b^2 - b^2 - (b-b)(b + b) = 0 - 0 = 0$$
\item Довести, що
\begin{itemize}
\item $\sin x \leq x,\ x > 0$
\item $e^x > 1 + x + \frac{x^2}{2},\ x > 0$
\end{itemize}

\begin{itemize}
\item $f(x) = x - \sin x$	. Треба довести, що $f(x) \geq 0$ для всіх $x > 0$. 
$$f'(x) = 1 - \cos x \geq 0 \Longrightarrow \textrm{$f(x)$ --- неспадна. Отже:}$$
$$\min f(x) = f(0)$$
$$\forall x \geq 0 \ \ f(x) \geq f(0) = 0$$
\item Треба довести, що $f(x) = e^x - 1 - x - \frac{x^2}{2} > 0,\ x > 0$.

$$f'(x) = e^x - 1 - x $$
Позначимо нову функцію $g(x)$ наступним чином:
$$g(x) = e^x - 1 - x$$
$$g'(x) = e^x - 1 > 0\ \forall x > 0$$
Це означає, що $g'(x) > 0 \Longrightarrow g(x)$ --- строго монотонно зростаюча на проміжку $(0, +\infty)$.

$$g(x) \in C(\mathbb{R})$$
$$g(x) > g(0) = 0$$
Отже $g(x) > 0\ \forall x > 0 \Longrightarrow f'(x) > 0 \ \forall x > 0$.
$$f(x) \in C(\mathbb{R}),\textrm{ а отже:}$$
$$f(x) > f(0) = 1 - 1 - 0 - 0 = 0 \ \forall x > 0$$
\end{itemize}
\item 
Довести, що 
$$\frac{2}{2x + 1} < \ln (1 + \frac{1}{x}),\ x > 0$$
Є два способи:
\begin{itemize}
\item 
$$f(x) = \ln (1 + \frac{1}{x}) - \frac{2}{2x + 1}$$
$$\lim_{x \to +\infty}f(x) = 0$$
Отже, лишилося лише довести, що $f'(x) < 0$ при $x > 0$. Це ми можемо зробити аналогічно з минулими прикладами.
\item Замінимо $x$ на іншу змінну $y$ наступним чином: 
$$y = \frac{1}{x} \in (0, +\infty)$$
$$\frac{2}{\frac{2}{y} + 1} < \ln(1 + y) \Longleftrightarrow \ln(1 + y) > \frac{2y}{2 + y}$$
$$f(y) = \ln(1 + y) - \frac{2y}{2 + y}$$
$$f(0) = 0$$
Отже, нам варто довети, що $f'(y) > 0$ при $y > 0$.
\end{itemize}
\end{enumerate}
\textbf{Означення. } Кажуть, що $f$ змінює знак з $+$ на $-$ в точці $x_0$, якщо:
$$\exists \varepsilon > 0:$$
$$\forall x \in (x_0 - \varepsilon, x_0)\ f(x) > 0$$
$$\forall x \in (x_0, x_0 + \varepsilon)\ f(x) < 0$$
Аналогічно означується зміна знаку з $-$ на $+$.

\vspace{3mm}

\textit{Наприклад,} 
\begin{itemize}
\item 
$x^3$ змінує знак з мінуса на плюс у точці $0$.  
\item 
$\sin x$ змінює знак з плюса на мінус у точці $\frac{\pi}{2}$.
\item
Функція $x^2 \sin (\frac{1}{x})$ в точці $x_0 = 0$ не змінює знак, бо ми не можемо визначити точне $\varepsilon$ з означення зміни знаку. Хоча, звичайно, ця функція не є знакосталою.
\end{itemize}

\vspace{2mm}

\textbf{Означення.} Точка $x_0$ називається стаціонарною, якщо $f'(x_0) = 0$.

\vspace{3mm}

\textbf{Теорема. (Перша достатня умова екстремуму)}.
Нехай $f \in D(I_{\varepsilon}(x_0))\setminus\{ x_0\}$ і $f \in C(\{ x_0\})$.

Якщо $f'$ в точці $x_0$ змінює знак з мінуса на плюс, то $f$ має в точці $x_0$ локальний мінімум, а якщо $f$ змінює знак з плюса на мінус, то в точці $x_0$ $f$ має локальний максимум.  

\textbf{Доведення.} $\forall x \in I_{\varepsilon}(x_0) \setminus \{ x_0\}$ розглянемо проміжок $[x_0, x]$. За теоремою Лагранжа:
$$\exists \xi \in (x_0, x):$$
$$f(x) - f(x_0) = f'(\xi) (x - x_0)$$
Якщо $x < x_0 \Longrightarrow x - x_0 < 0$, при цьому на проміжку $(x, x_0)$ $f'$ примає значення менше $0$, отже $f'(\xi) < 0$.
$$f'(\xi) < 0,\ x - x_0 < 0 \Longrightarrow f(x) - f(x_0) = f'(\xi) (x - x_0) > 0$$. Отже, $f(x) > f(x_0)$. 

Аналогічно доводиться, що на проміжку $(x_0, x_0 + \varepsilon)$ $f(x) > f(x_0)$.

Отже, $\forall x \in I_{\varepsilon}(x_0) \setminus \{x_0 \} \ f(x) - f(x_0) > 0$, тобто $x_0$ --- точка локального максимуму.

\textit{Цікавий випадок:}
До функції
$$f(x) = \begin{cases} x^2 \sin \frac{1}{x}, & x\neq 0 \\ 
						0, & x = 0\end{cases}$$
Дану теорему при $x_0 = 0$ застосувати не можемо, бо не можемо сказати, як вона змінює знак у точці $0$.

\textit{До речі} лектор сказав, що ця функція це якийсь \textbf{жарт}, тож її на модулі \textbf{не буде}.

\vspace{3mm}

\textbf{Теорема. (Друга достатня умова екстремуму).} Нехай в стаціонарній точці $x_0 \ \exists f''(x_0)$.
\begin{itemize}
\item Якщо $f''(x_0) > 0$, то в точці $x_0$ функція $f$ має локальний мінімум.
\item Якщо $f''(x_0) < 0$, то в точці $x_0$ функція $f$ має локальний максимум. 
\end{itemize} 

\textbf{Доведення.} Доведемо для випадку $f''(x_0) > 0$. Випадок $f''(x_0) < 0$ доводиться аналогічно.
$$f''(x_0) = \lim_{x \to x_0}\frac{f'(x) - f'(x_0)}{x - x_0} > 0$$
$$\textrm{Якщо для деякої функції $\Phi$}\lim_{x\to x_0} \Phi (x) > 0, \textrm{ то}$$
$$\exists \varepsilon : \forall x \in I_{\varepsilon}(x_0)\  \Phi(x) > 0$$
У нашому випадку виходить, що:
$$\exists \varepsilon : \forall x \in I_{\varepsilon}(x_0)\ \frac{f'(x) - f'(x_0)}{x - x_0} > 0$$
\begin{itemize}
\item Якщо $x > x_0$, то $f'(x) - f'(x_0) > 0$
\item Якщо $x < x_0$, то $f'(x) - f'(x_0) < 0$
\end{itemize}
Оскільки $x_0$ --- стаціонарна точка, то $f'(x_0) = 0$, отже:
\begin{itemize}
\item Якщо $x > x_0$, то $f'(x) > 0$
\item Якщо $x < x_0$, то $f'(x) < 0$
\end{itemize}
Отже, $f'$ змінює знак з $-$ на $+$ у точці $x_0$. З попередньої теореми отримуємо, що в $f$ має у точці $x_0$ локальний мінімум.

\vspace{3mm}

\textit{Домашня робота.} Лектор сказав, щоб ми самі розібрали Третю достаню умову екстремуму, та усе що пов'язане з крайовими екстремумами.

\end{document} 
