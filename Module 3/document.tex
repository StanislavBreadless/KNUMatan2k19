\documentclass[12pt]{report}
\usepackage[utf8x]{inputenc}
\usepackage[russian]{babel}
\usepackage{amssymb}
\usepackage{amsmath}
\usepackage[makeroom]{cancel}

\title{Конспекти лекцій з математичного аналізу Анікушина А.В. Модуль 3.}
\author{Автор текста @bezkorstanislav \\ Если есть ошибки, пишите ему в телеграм \\  \small{Aфтар выражает благодарность @vic778 за многочисленные поправки}}

\date{October 2019}

\begin{document}

\maketitle

\begin{center}
\textbf{\LargeДиференціальне числення функції однієї змінної}
\end{center}

\begin{center}
\textbf{\large{Означення похідної. Основні правила диференційонування}}
\end{center}

Нехай $f : \mathbb{R} \to \mathbb{R}$ --- деяка функція однієї змінної, $x_0 \in D_f,\ x_0 \in (D_f)^{'}$ .

\textbf{Означення.} Якщо $\exists \lim\limits_{x \to x_0} \frac{f(x) - f(x_0)}{x - x_0}$, то функція $f$ називається 
\textbf{диференційованою} в точці $x_0$, а сама границя називається \textbf{похідною} в точці $x_0$. І позначається $f'(x)$ або $\frac{d f(x)}{d x}$.

\vspace{3mm}

\textit{Зауваження 1.} Для функції однієї змінної ми ототожнили диференційованість та існування похідної.

\textit{Зауваження 2.} $$x - x_0 = \Delta x$$
$$f'(x_0) = \lim_{\Delta x \to 0} \frac{f(x_0 + \Delta x) - f(x_0)}{\Delta x}$$
$$f(x_0 + \Delta x) - f(x_0) = \Delta f(x)$$
Отже, похідна дорівнює відношенню зміни приросту функції до приросту аргументу, що породжує цей приріст.

\textit{Зауваження 3.} 
$$f'(x_0) = \lim_{x \to x_0} \frac{f(x) - f(x_0)}{x - x_0} \Longleftrightarrow \frac{f(x) - f(x_0)}{x - x_0} = f'(x_0) + o(1)$$
$$f(x) - f(x_0) = f'(x_0)(x-x_0) + o(x-x_0)$$
Отже, якщо має місце рівність:
$$f(x) - f(x_0) = A (x - x_0) + o(x - x_0) \Longrightarrow \exists f'(x_0) = A$$

\textbf{Теорема. (Необхідна умова диференційованості).} Функція $f$ є диференційованою в $x_0$ тільки тоді, коли $f$ --- неперервна в точці $x_0$.

\textbf{Доведення.} Щоб існувала похідна треба, щоб
$$f(x) - f(x_0) \to 0,\to x \to x_0$$.
З цього випливає, що $f$ неперервна в точці $x_0$.

\textbf{Теорема. (Диференційованість композиції функцій).}  Нехай дано функції $f$ і $g$. точка $x_0 \in D_{f \circ g},\ x_0 \in (D_{f \circ g})'$.

Якщо $g$ диференційована в точці $x_0$, а $f$ диференційована в точці $y_0 = g(x_0)$, то $f \circ g$ диференційована в точці $x_0$ і має місце рівність:
$$(f \circ g)' (x_0) = f'(g(y_0)) g'(x_0)$$

\textbf{Доведення.} 
$$(f \circ g) (x) - (f \circ g) (x_0) = f(g(x)) - f(g(x_0)) = f(y) - f(y_0) =$$ 
$$= f'(y_0)(y - y_0) + o(y - y_0) = f'(y_0)(g(x) - g(x_0)) + o(y - y_0) = $$
$$= f'(y_0)(g'(x_0)(x - x_0) + o(x - x_0)) + o(y - y_0) = $$
$$= f'(y_0)(g'(x_0)(x - x_0)(1 + o(1))) + o(g(x) - g(x_0)) = $$
$$= f'(y_0)(g'(x_0)(x - x_0)(1 + o(1))) + o(g(x_0)(x - x_0) + o(x - x_0))$$

Отже,

$$\lim_{x \to x_0} \frac{(f \circ g)(x) - (f \circ g)(x_0)}{x - x_0} = $$
$$=\lim_{x \to x_0}\frac{f'(y_0)(g'(x_0)(x - x_0)(1 + o(1))) + o(g(x_0)(x - x_0) + o(x - x_0))}{x - x_0} = $$
$$=\lim_{x \to x_0}f'(y_0)(g'(x_0)(1 + o(1))) + o(g(x_0) + o(1)) = $$
$$=\lim_{x \to x_0}f'(y_0)g'(x_0)(1 + o(1)) + o(1) = f'(y_0)g'(x_0)$$

\textbf{Теорема. (Лінійність похідної).} Нехай $f$ і $g$ --- диференційовані в точці $x_0$, $\alpha,\beta \in \mathbb{R}$, то:
$$(\alpha f + \beta g)' (x_0) = \alpha f'(x_0) + \beta g'(x_0)$$

\textbf{Доведення.} 

$$\lim_{x \to x_0} \frac{(\alpha f + \beta g)(x) - (\alpha f + \beta g)(x_0)}{x - x_0} = $$
$$\lim_{x \to x_0} \frac{ \alpha f(x) - \alpha f(x_0)}{x - x_0} + \frac{\beta g(x) - \beta g (x_0)}{x - x_0} = $$
$$\lim_{x \to x_0} \alpha \frac{ f(x) - f(x_0)}{x - x_0} + \lim_{x \to x_0}\beta \frac{g(x) - g (x_0)}{x - x_0}  = \alpha f'(x_0) + \beta g'(x_0)$$




 

\end{document} 
