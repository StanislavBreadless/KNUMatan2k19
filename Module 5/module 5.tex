\documentclass[12pt]{report}
\usepackage[utf8x]{inputenc}
\usepackage[russian]{babel}
\usepackage{amssymb}
\usepackage{amsmath}
\usepackage[makeroom]{cancel}
\usepackage{mathtext} 
\usepackage{relsize}
\usepackage{scalerel}
\usepackage[usestackEOL]{stackengine}
\title{Конспекти лекцій з математичного аналізу Анікушина А.В. Модуль 4.}
\author{Автор тексту @vic778 \\ Якщо знайшли помилки, пишіть мені в телеграм \\  \small{A special thanks to @bezkorstanislav without whom these lecture notes would never have been created}}

\date{February 2020}

\begin{document}
	\maketitle

\begin{center}
	\textbf{\LargeІнтеграл Рімана} 
\end{center}
 
 Нехай f: [a,b]	$\rightarrow \mathbb{R}$. Розіб'ємо [a,b] на n частин точками 
 
 a=$x_0<x_1<x_2<...<x_n=b$. Сукупність  $\{x_0, x_1, ..., x_n\}$ = P назвемо \textit{ розбиттям} [a,b]. Розглянемо довжини $\Delta x_i = x_i - x_{i-1},     i = 1, ..., n.$
Число $\textit{diam} (P)$ = max$ \Delta x_i$ назвемо $\textit{діаметром}$ розбиття.

Тепер на кожному проміжку $[x_{i-1}, x_i]$ оберемо довільну точку  

$\xi \in [x_{i-1}, x_i]$. Множину $\xi = \{\xi_1, \xi_2, ..., \xi_n\}$ назвемо  \textit{сукупністю} проміжних точок, що віповідає розбиттю P. 

Тепер утворимо таку суму \[  S_p (f, \xi)  = \sum\limits_{i=1}^n f(\xi_i)\Delta x_i \]
$ S_p (f, \xi)   $ називається інтегральною сумою Рімана для функції f на відрізку [a,b], що побудована за розбиттям P і сукупністю проміжних точок $ \xi $.

$ f(\xi_1)(x_1 - x_0)+ f(\xi_2)(x_2 - x_1)+ f(\xi_3)(x_3 - x_2) +f(\xi_4)(x_4- x_3) $ 

(Інтегральна сума дорівнює сумі площ прямокутників).

\textit{Означення} Число $I$ називається $\textit{інтегралом Рімана} $ від функції  $ f $
на [a,b], якщо $ \forall \varepsilon > 0 $   $ \exists \delta > 0$   $ diam (P) < \delta$ $\Rightarrow  $ $ \forall \xi $ $ |S_p(f\xi) - I|  < \varepsilon$

\vspace{5 mm} 

\textbf{Лема. (Необхідна умова інтегровності за Ріманом).} Якщо функція $ f $ інтегровна за Ріманом, то $ f $ - обмежена на [a,b].

Якщо $ f $ - необмежена, то $ \forall n $  $ \exists y_n $ : $ f(y_n) > n$ $ y_nk \rightarrow y $. Тоді при деякому $ \xi $ $ S_p (f, \xi)  \rightarrow \infty$.

\vspace{5 mm} 
Сукупність всіх функцій, інтегрованих за Ріманом, позначають $ R([a,b]) $.

\begin{center}
	\textbf{\Large Чи пов'язані між собою інтеграли Рімана та Ньютона-Лейбніца?} 
\end{center}

\textbf{Теорема 1.} Нехай  $ f $ - інтегровна за Ньютоном-Лейбніцом. Тоді $ \forall P  $
$ \exists \xi : $ $\mathlarger\int\limits_ a^b f(x)dx$ = $ S_p (f, \xi) $

\textbf{Доведення} $\mathlarger\int\limits_ a^b f(x)dx =  \sum\limits_{k=0}^{n-1}\mathlarger\int\limits_{x_k}^{x_{k+1}} f(x)dx = \sum\limits_{k=0}^{n-1}(F(x_{k+1}) - F(x_k)) =$

$=  \sum\limits_{k=0}^{n-1} F'(\xi_{k+1}) (x_{k+1} - x_k) = \sum\limits_{k=0}^{n-1} f(\xi_{k+1})(x_{k+1} - x_k)$

\vspace{5 mm} 
\textit{Тобто, за кожним інтегралом Ньютона-Лейбніца стоїть інтеграл Рімана.}

\vspace{5 mm} 

\textbf{Теорема 2} Якщо існують інтеграли Рімана та Ньбтона-Лейбніца, то вони співпадають.

\textbf{Доведення} Нехай $ I $ - інтеграл Рімана. Тоді $ \forall \varepsilon > 0 $  $\exists \delta > 0  $ $ \forall P: diam (P)  < \delta , $  $ \forall \xi |S_p(f, \xi) - I| < \varepsilon.$ Тепер 
$ \forall P$ $ \exists \xi_0  $ $ \mathlarger\int\limits_a^b f(x) dx - S_p (f, \xi) = 0.$
 Якщо $ diam(P)< \delta $, то |$\mathlarger\int\limits_a^b f(x) dx  - I  $| < $ \varepsilon. $
 
Отже, $\mathlarger\int\limits_a^b f(x) dx   = I$.

\vspace{5 mm} 
\textit{Зауваженняя} Можна показати, що всі неперервн функції інтегровні за Ріманом.

\vspace{5 mm} 

$ \overline S_p(f) = \sum M_i (x_{i+1} - x_{i}) ,$ де $ M_i = supf(x)  x_i [x_i, x_{i+1}]$
називають \textit{верхньою інтегральною сумою Дарбу.}

Аналогічно, $ \underline S_p(f) = \sum M_i (x_{i+1} - x_{i}) ,$ де $ M_i = inf(x)  x_i [x_i, x_{i+1}]$ називають \textit{нижньою інтегральною сумою Дарбу.}

\vspace{3 mm} 
Зрозуміло, що $ \forall \xi  $ $ \underline S_p(f) \leq S_p (f, \xi) \leq  \overline S_p(f)$


\vspace{5 mm}

Нехай $P$ --- розбиття. $P_1 = P \cup \{ x^{*}\}$.

$$\overline S_p(f) = \sum_{k = 1}^m M_{k} \Delta x_k + M_{m+1}\Delta x_{m+1} + \sum_{k = m+2}^n M_k \Delta x_k$$

$$\overline S_{p_1}(f) = \sum_{k = 1}^m M_{k} \Delta x_k + \sup_{x \in [x_{m}; x^{*}]} f(x) (x^* - x_m) + \sup_{x \in [x^{*}; x_{m+1}]} f(x) (x_{m+1} - x^*) + \sum_{k = m+2}^n M_k \Delta x_k$$
$$A \subset B \Longrightarrow \sup_{A} f \leq \sup_{B} f$$
$$\sup_{x \in [x_{m}; x^{*}]} f(x) \leq M_{m+1}$$
$$\sup_{x \in [x^{*}; x_{m+1}]} f(x) \leq M_{m+1}$$

$$\overline S_{p_1}(f) \leq \sum_{k = 1}^m M_{k} \Delta x_k + M_{m+1}\Delta x_{m+1} + \sum_{k = m+2}^n M_k \Delta x_k$$
$$\overline S_p \geq \overline S_{p+1}$$
Додаючи точку до розбиття, верхня (з супремами) інтегральна сума може зменшитися або залишитис такою ж.

Тому зрозуміло, що 
$$P_1 \subset P_2 \Longrightarrow \overline S_{p_1} \geq \overline S_{p_k}, \underline S_{p_1} \leq \underline S_{p_2}$$

\textbf{Наслідок:}

$$\forall P_1, P_2 \overline S_{P_1}(f) \geq \underline S_{P_2}(f)$$ 

Розглянемо розбиття $P = P_1 \cup P_2$.
$$\overline S_{P_1} \geq \overline S_{p} \textrm{(з попередньої теореми $P_1 \subset P_2 \Longrightarrow \overline S_{P_1} \geq \overline S_{p}$)}$$
$$\underline S_{P} \geq \underline S_{P_2}$$
$$\overline S_{P_1} \geq \overline S_{P} \geq \underline S_{P} \geq \underline S_{P_2}$$
$$\overline S_{P_1} \geq \underline S_{P_2}$$
Що й треба було довести.

\vspace{3mm}

Розглянемо всі можливі верхні суми. $\overline S_{P} (f)$. Ця множина є обмеженою знизу (Необхідна умова інтегрованості за Ріманом). 
Тому існує $\inf_{P} \overline S_{P} (f) = \overline \int f dx$. Назвемо це число верхнім інтегралом Дербу. Аналогічно 
нижній інтеграл Дербу $\underline \int f dx = \sup_{P} \underline S_{p} (f)$.

\vspace{3mm}

Розглянемо нерівність $\overline S_{P_1} (f) \geq S_{P_2} (f)$. 
Зафіксуємо $P_2$. Тоді 
$$\forall P_1 \overline S_{P_1} (f) \geq \underline S_{P_2} \Longrightarrow \inf \overline S_{P_1} (f) \geq \underline S_{P_2} (f)$$
(Всі елементи множини $>$ за фіксоване число $\Longrightarrow$ $>$ за інфінум.)

Аналогічно для $P_2$ та супремума:

Зафіксуємо $P_1$. Тоді:
$$\forall P_2 \ \underline S_{P_2} (f) \leq \overline S_{P_1} (f) \Longrightarrow \sup \overline S_{P_1} (f) \leq \underline S_{P_1} (f)$$
Звідcи $\overline \int fdx \geq \underline f dx$.

\textbf{Означення}. Функція $f$ називається інтегровною за Дарбу, якщо $\overline \int f dx = \underline \int fdx$. (Найкраше наближення зверху = найкраще наближення знизу).

\textbf{Теорема. (Критерій інтегрованості за Дарбу)}. Функція $f$ є інтегровною за Дарбу тоді й тільки тоді, коли:
$$\forall \varepsilon > 0 \ \exists P : |\overline S_{p}(f) - \underline S_{P} (f)| < \varepsilon$$
(Можна підібрати число, для якого верхня і нижня інтегральні суми відрізняються на мале число)

\textbf{Доведення}

\begin{itemize}
\item $\Longleftarrow.$
$$\underline S_{P} (f) fdx \leq \overline \int fdx \leq \overline S_{P} (f)$$ 
$$\underline S_{P} (f) \leq \underline \int fdx \leq \overline \int f dx \leq \overline S_{P} (f)$$
$$\varepsilon \geq \overline S_{P} (f) - \underline S_{P} (f) \geq \overline \int f dx - \underline \int f dx \geq 0$$

Отже, $\overline \int f dx - \underline f dx = 0$.

\item $\Longrightarrow.$

Зафіксуємо $\varepsilon > 0$. Тоді супремум $\inf \overline S_{P} (f) = \overline \int f dx$ є точкою дотику 
в будь-якому $\varepsilon$-околі множини.
$$\exists P_{1} \ \overline \int f dx \leq \overline S_{P_1} (f) \leq \overline \int fdx + \frac{\varepsilon}{2}$$
$$\exists P_{2} \ \underline \int f dx \leq \underline S_{P_2} (f) \leq \underline \int fdx $$

Розглянемо $P = P_1 \cup P_2$. Збільшуємо розбиття, збільшуємо точність, верхня інтегральна сума збільнується (або залишається такою ж).

$$0 \leq \overline S_{P} (f) - \underline S_{P} (f) \leq \overline S_{P_1} - \underline S_{P_2} (f) < \varepsilon$$
Що і треба було довести.
\end{itemize}

\vspace{5mm}

\textbf{Теорема.} Функція $f$ є інтегрованою за Дарбу тоді й тільки тоді, коли $f$ інтегровна за Ріманом і їх інтеграли співпадають.

\vspace{3mm}

\textit{Приклад 1.}
$$0 = x_0,\ x_k = \frac{1}{n},\ x_n = 1$$
$$\overline S_{P} (f) = \sum_{k=1}^n M_k \cdot \Delta x_k = \sum_{k=1}^n \frac{k}{n} \cdot \frac{1}{n}$$
$$M_k = \sup_{[x_{k-1}, x_k]} x = x_k = \frac{k}{n}$$
$$\underline S_{P} (f) = \sum_{k = 1}^n M_k \Delta x_k = \sum_{k=1}^n x_k + \Delta x_k = \sum_{k=1}^{n} \frac{k-1}{n} \cdot \frac{1}{n}$$
$$\overline S_{P}(f) - \underline S_{P} (f) = \frac{1}{n}$$
$\frac{1}{n}$ може бути як завгодно мале, а тому $x \in R([0,1])$ (інтегрована за Ріманом).
\vspace{3mm}

\textit{Приклад 2.}
$$D(x) =  \begin{cases} 1 &, x \in \mathbb{Q} \\
						  0 &, x \in \mathbb{R} \setminus \mathbb{Q} \end{cases}$$
Функцію розглядаємо на проміжку $[0,1]$.

$$\overline S_{P} (D) = \sum_{k=1}^n \sup_{x \in [\ldots]} D(x) = \sum_{k=1}^n \Delta x_k = 1$$
$$\underline S_{P} (D) = \sum_{k=1}^n( \inf_{x \in [\ldots]} (D(x)) \cdot  \Delta x_k) = 0$$
$$\forall p : \overline S_{p}(D) - \underline S_{p} (D) > \frac{1}{13} =  \varepsilon$$
Отже, $D \notin R([0,1])$.

\begin{center}
	\textbf{\LargeМножини Лебегової міри нуля} 
\end{center}

\textbf{Означення} Множина $A \subset \mathbb{R}$ має міру нуль, якщо $\forall \varepsilon > 0 \ \exists$ не більш як зліченна кількість $ (\alpha_1,\beta_1), (\alpha_2,\beta_2), \ldots$ такі, що:

\begin{enumerate}

\item $A \subset \bigcup_{k=1}^{\infty} (\alpha_k, \beta_k)$ (Множину можна покрити інтервалами, що є завгодно малими)

\item $\sum (\beta_k - \alpha_k) < \varepsilon$

\end{enumerate}

\textit{Приклад}:

\begin{enumerate}

\item $A = \{ x_0\}$
\item $A = \{ x_1, x_2, \ldots, x_n\}$
\item $A = (0; \frac{1}{2})$ При $\varepsilon < \frac{1}{2}$ покритий не існує. (Якщо є неперервна множина, з більше, ніж однієї точки), $A$ не має точки нуль).
\item $A = \{ x_1, x_2, \ldots\}$
\end{enumerate}

\textbf{Властивості:}

\begin{enumerate}

\item Якщо $A$ -- зліченна, або обмежена, то $A$ має міру нуль.
\item Якщо $\exists \alpha, \beta : (\alpha, \beta) \subset A \Longrightarrow A$ не має міру нуль.
\item Незліченні (континуальні) множини теж мають міру нуль.
\item Якщо $A_1, A_2, \ldots$ мають міру нуль, то їх об'єднання теж має міру нуль.  

\end{enumerate}

\textbf{Приклад:}

\begin{itemize}

\item $Q \cap [0,1]$ -- зліченна, тож має міру нуль.
\item $(\mathbb{R} \setminus \mathbb{Q}) \cap [0,1]$. Якщо припустити, що $(\mathbb{R} \setminus \mathbb{Q}) \cap [0,1]$ 
має міру нуль, то за властивістю $4$:

$(\mathbb{Q}\cap[0,1]) \cup (\mathbb{R} \setminus \mathbb{Q}) \cap [0,1] = [0,1]$ теж має міру 0. Але це суперечить властивості $2$.

\end{itemize}
 

\textbf{Критерій Лебега інтегровності за Ріманом}

Нехай $f$ -- обмежена на $[a,b]$. Тоді $f \in R([a,b])$ тоді й тільки тоді, коли множина точок розриву функції $f$ на $[a,b]$ має міру $0$.
(Тобто коли точок розриву не більше, ніж зліченна кількість).

\textbf{Приклади:}

\begin{enumerate}
\item D(x) = $\begin{cases} 1 &, x \in \mathbb{Q} \\ 0 &, x \notin \mathbb{Q} \end{cases}$ на проміжку $[0,1]$.

$E_{D} = [0,1]$ -- не має міру $0$ за властивостю $2$.

Отже, за критерієм $D \notin R([0,1])$.

\item $f(x) = \frac{x - \frac{1}{2}}{|x - \frac{1}{2}|} = \begin{cases} 1 &, x > \frac{1}{2} \\ -1 &, x < \frac{1}{2} \end{cases}$

$E_f = \{\frac{1}{2}\}$ -- множина точок розриву має міру нуль. Отже, $f(x) \in R([0,1])$.

\item Функція Рімана:

$$f(x) = \begin{cases} 0 &, x \in \mathbb{R} \setminus \mathbb{Q} \\ 
					   \frac{1}{n} &, x = \frac{m}{n}, \textrm{НСД($m$,$n$) = 1}
		 \end{cases}$$
$E_f = \mathbb{Q} \cap [0,1]$. Отже, $f$ інтегрована.

\item $f(x) = \ln x$

$E_f = \{0\}$ -- точки розриву. Але $f(x)$ необмежена, тому не є інтегрованою за Ріманом на $[0,1]$.

\end{enumerate}
 
\end{document}  